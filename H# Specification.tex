\documentclass{article}

% Packages
\usepackage[utf8]{inputenc}
\usepackage[english]{babel}
\usepackage{amsmath}
\usepackage{latexsym}
\usepackage{amsfonts}
\usepackage[normalem]{ulem}
\usepackage{array}
\usepackage{amssymb}
\usepackage{graphicx}
\usepackage{fancyhdr}
\usepackage{siunitx}
\usepackage{algorithm}
\usepackage[noend]{algpseudocode}
\usepackage{listings}
\usepackage{float}
\usepackage[hidelinks]{hyperref}
\usepackage{caption}
\usepackage{logicproof}
\usepackage{longtable}
\usepackage{syntax}
\usepackage{lastpage}
\usepackage[a4paper,left=1in,right=1in,top=1in,bottom=1in]{geometry}
\usepackage{pdfpages}

%%% Hyperlink setup
\hypersetup {
	%linktoc = none
}

%%% Gramar environment
\setlength{\grammarparsep}{6pt plus 1pt minus 1pt} % increase separation between rules
\setlength{\grammarindent}{12em} % increase separation between LHS/RHS 
\let\syntleft\relax
\let\syntright\relax

%%% Code environment
\usepackage{listings}
\usepackage{xcolor}
\usepackage{mathpartir}
\definecolor{commentsColor}{rgb}{0.497495, 0.497587, 0.497464}
\definecolor{keywordsColor}{rgb}{0.000000, 0.000000, 0.635294}
\definecolor{stringColor}{rgb}{0.558215, 0.000000, 0.135316}

\lstset{
  	basicstyle=\ttfamily\small,                   % the size of the fonts that are used for the code
  	breakatwhitespace=false,                      % sets if automatic breaks should only happen at whitespace
  	breaklines=true,                              % sets automatic line breaking
  	frame=tb,                                     % adds a frame around the code
  	commentstyle=\color{commentsColor}\textit,    % comment style
  	keywordstyle=\color{keywordsColor}\bfseries,  % keyword style
  	stringstyle=\color{stringColor},              % string literal style
  	numbers=left,                                 % where to put the line-numbers; possible values are (none, left, right)
  	numbersep=5pt,                                % how far the line-numbers are from the code
  	numberstyle=\tiny\color{commentsColor},       % the style that is used for the line-numbers
  	showstringspaces=false,                       % underline spaces within strings only
  	tabsize=2,                                    % sets default tabsize to 2 spaces
  	language=c++
}

% Titles
\title{H\# Official Documentation}
\author{The official specification for H\#}
\date{}

% Page styling
\fancyhf{}
\renewcommand{\headrulewidth}{0pt}
\rfoot{Page \thepage \hspace{1pt} of \pageref{LastPage}}

% Custom commands
\newcommand{\newsection}[1]{
	\phantomsection
	\addcontentsline{toc}{section}{#1}
	\section*{\centering #1}
}
\newcommand{\newsubsection}[1]{
	\phantomsection
	\addcontentsline{toc}{subsection}{#1}
	\subsection*{#1}
}
\newcommand{\newsubsubsection}[1]{
	\phantomsection
	\addcontentsline{toc}{subsubsection}{#1}
	\subsubsection*{#1}
}

%% Semantic commands
\newcommand{\envall}{\rho, \mu, \phi, \kappa, \sigma \vdash}
\newcommand{\envallm}[1]{\rho, \mu, \phi, \kappa, \sigma #1 \vdash}
\newcommand{\typemodel}{\theta, \gamma, \eta \vdash}
\newcommand{\typeis}[1]{\ \texttt{:} \ #1}
\newcommand{\subtype}[2]{#1 \ \texttt{<:} \ #2}
\newcommand{\rspace}{ \ \ \ \ }

%% Grammar commands
\newcommand{\gtext}[1]{<$#1$>}
\newcommand{\glit}[1]{\texttt{#1}}

\begin{document}
\pagestyle{empty}
\pagenumbering{gobble}
\maketitle
\tableofcontents
\clearpage
\pagestyle{fancyplain}
\setcounter{page}{1}
\pagenumbering{arabic}
\newsection{Abstract}
This document is the official document for the H-Sharp (H\#) programming language. The H stands for hybrid - as this is a multi-paradigm programming language heavily inspired by C\# and Scala. This document will only contain the grammar of the language, the operational semantics as well as the type semantics of the language. The operational semantics may be explained with code samples - but obvious uses will not be explained.\\\\
This document will also contain the byte-instruction semantics. The official compiler, is written in C\# for productivity purposes while the Virtual Machine is implemented using C++. At no point will this document be documenting the internal processes of those applications.
\newpage
\newsection{Grammar}
The offical H\# grammar. Note: $Id$ $\in$ \textsc{VarEnv} and $TypeId$ $\in$ \textsc{TypeEnv}. Both refer to the same grammatical definition, defined by the regular expression:
$$(\glit{_}|[\glit{a}-\glit{Z}])^+(\glit{_}|[\glit{0}-\glit{9}]|[\glit{a}-\glit{Z}])^*$$
If multiple elements can occur - and it'd be convenient, the element may be suffixed with 'n' to show it may contain n of such elements. $n\in\{0,1,2,\dots\}$. The notation $e_0\glit{,}\dots\glit{,}e_n$ represents the one to nth element with a specific separator. The full grammar is defined as follows:
\begin{grammar}

	\gtext{CompileUnit} ::= $CompileUnitElement$

	\gtext{CompileUnitElement} ::= $CompileUnitElement$ $CompileUnitElement$
	\alt $Directive$ | $Scope$ | $ScopeElement$ | $Declaration$

	\gtext{Scope} ::= \glit{\{} $ScopeElement$ \glit{\}}
	
	\gtext{ScopeElement} ::= $ScopeElement$ $ScopeElement$
	\alt $Expr$\glit{;} | $Statement$	| $VarDeclaration$\glit{;}
	
	\gtext{Expr} ::= $Expr$ | \glit{(}$Expr$\glit{)} | $Scope$ | $LambdaExpr$
	\alt $Expr$ $BinaryOp$ $Expr$ | $UnaryOp$ $Expr$ | $Id$ $UnaryOp$
	\alt $Expr$ \glit{?} $Expr$ \glit{:} $Expr$
	\alt $Id$ | $Name$ | \glit{this} | \glit{base}
	\alt $Expr$\glit{(}$Argument$\glit{)} | $Expr$\glit{[}$Expr$\glit{]}
	\alt \glit{new} $TypeId$\glit{(}$Argument$\glit{)} | \glit{new} $TypeId$\glit{[}$Argument$\glit{]}
	\alt $Expr$\glit{.}$Id$ | $Expr$\glit{?.}$Id$ 
	\alt $Expr$\glit{.}$Id$\glit{(}$Argument$\glit{)} | $Expr$\glit{?.}$Id$\glit{(}$Argument$\glit{)}
	\alt $Expr$ \glit{is} $TypeID$ | $Expr$ \glit{is null} $Expr$ \glit{is} $TypeID$ $Id$
	\alt $Expr$ \glit{is} $TypeID$ $Id$ \glit{where} $Expr$
	\alt $Expr$ \glit{is not} $TypeID$ | $Expr$ \glit{is not null}
	\alt $Expr$ \glit{is not} $TypeID$ $Id$ \glit{where} $Expr$ % if type and condition is false
	\alt $Expr$ \glit{as} $TypeId$
	\alt \glit{(}$TypeID$\glit{)} $Expr$
	\alt \glit{sizeof($TypeId$}\glit{)} | \glit{addressof($Id$}\glit{)} | \glit{typeof($Expr$}\glit{)}
	\alt $Assignment$
	\alt $Literal$
	
	\gtext{LambdaExpr} ::= \glit{(}$Param$\glit{) =>} $Expr$
	
	\gtext{Directive} ::= \glit{type} $Id$ \glit{=} $TypeId$\glit{;}
	\alt \glit{using} $Name$\glit{;} | \glit{using} $TypeId$ \glit{from} $Name$\glit{;}
	\alt $NamespaceDecl$

	\gtext{NamespaceDecl} ::= \glit{namespace} $Name$ $Scope$
	
	\gtext{Statement} ::= $Assignment$\glit{;} | $ControlStatement$ | $MatchStatement$\glit{;}
	\alt $TryCatchStatement$
	
	\gtext{ControlStatement} ::= \glit{if} $Expr$ $Scope$
	\alt \glit{if} $Expr$ $Scope$ \glit{else} $Scope$
	\alt \glit{if} $Expr$ $Scope$ \glit{else if} $Expr$ $Scope$
	\alt \glit{if} $Expr$ $Scope$ \glit{else if} $Expr$ $Scope$ \glit{else} $Scope$
	\alt \glit{while} $Expr$ $Scope$
	\alt \glit{do} $Scope$ \glit{while} $Expr$\glit{;}
	\alt \glit{for} \glit{(}$Assignment$\glit{;} $Expr$\glit{;} $Expr$\glit{)} $Scope$
	\alt \glit{for} \glit{(}$VarDeclaration$\glit{;} $Expr$\glit{;} $Expr$\glit{)} $Scope$
	\alt \glit{for} \glit{(}$TypeId$ $Id$ \glit{:} $Expr$\glit{)} $Scope$
	\alt \glit{throw} $TypeID$\glit{(}$Argument$\glit{)}
	\alt \glit{return} $Expr$\glit{;}
	\alt \glit{break;}
	
	\newpage	
	
	\gtext{MatchStatement} ::= $Expr$ \glit{match \{} $MatchCase$ \glit{\}} 
	
	\gtext{MatchCase} ::= $MatchCase$\glit{,} $MatchCase$
	\alt \glit{case} $Literal$ \glit{=>} $Expr$ | \glit{case} $TypeId$ \glit{=>} $Expr$ 
	\alt \glit{case} $TypeId$ $Id$ \glit{=>} $Expr$ | \glit{case} $TypeId$ $Id$ \glit{when} $Expr$ \glit{=>} $Expr$ 
	\alt \glit{case (}$MatchCaseId$\glit{) =>} $Expr$
	\alt \glit{case (}$MatchCaseId$\glit{) when} $Expr$ \glit{=>} $Expr$ 
	\alt \glit{case} $TypeId$ \glit{(}$MatchCaseId$\glit{) =>} $Expr$
	\alt \glit{case} $TypeId$ \glit{(}$MatchCaseId$\glit{) when} $Expr$ \glit{=>} $Expr$ 
	\alt \glit{case} \glit{_} \glit{=>} $Expr$
	
	\gtext{MatchCaseId} ::= $MatchCaseId$\glit{,} $MatchCaseId$
	\alt $Id$ | \glit{_}

	\gtext{TryCatchStatement} ::= \glit{try} $Scope$ \glit{catch} \glit{(}$TypeId$ $Id$\glit{)} $Scope$
	\alt \glit{try} $Scope$ \glit{catch} \glit{(}$TypeId$ $Id$\glit{) when} $Expr$ $Scope$  
	\alt \glit{try} $Scope$ \glit{catch} \glit{(}$TypeId$ $Id$\glit{)} $Scope$ \glit{finally} $Scope$
	\alt \glit{try} $Scope$ \glit{catch} \glit{(}$TypeId$ $Id$\glit{) when} $Expr$ $Scope$ \glit{finally} $Scope$

	\gtext{Assignment} ::= $Id$ \glit{=} $Expr$
	\alt $Id$ \glit{+=} $Expr$ | $Id$ \glit{-=} $Expr$
	\alt $Id$ \glit{*=} $Expr$ | $Id$ \glit{/=} $Expr$
	\alt $Id$ \glit{\&=} $Expr$ | $Id$ \glit{|=} $Expr$
	\alt $Id$ \glit{\%=} $Expr$
	
	\gtext{Declaration} ::=  $VarDecl$ | $FuncDecl$ % Variable declaration
	\alt $ClassDecl$ | $StaticClassDecl$ |  $TraitDecl$ % Ref types
	\alt $UnionDecl$ | $EnumDecl$ | $StructDecl$ % Value types
	\alt $TraitUniversal$ % Other
	
	\gtext{ClassDecl} ::= $Modifier$ \glit{class} $Id$ $ClassBody$
	\alt $Modifier$ \glit{class} $Id$ \glit{:} $ParamType$ $ClassBody$
	\alt $Modifier$ \glit{class} $Id$\glit{(}$Param$\glit{)} $ClassBody$
	\alt $Modifier$ \glit{class} $Id$\glit{(}$Param$\glit{) :} $ParamType$ $ClassBody$
	
	\gtext{StructDecl} ::= $Modifier$ \glit{struct} $Id$ $ClassBody$
	\alt $Modifier$ \glit{struct} $Id$ \glit{:} $ParamType$ $ClassBody$
	\alt $Modifier$ \glit{struct} $Id$\glit{(}$Param$\glit{)} $ClassBody$
	\alt $Modifier$ \glit{struct} $Id$\glit{(}$Param$\glit{) :} $ParamType$ $ClassBody$
	
	\gtext{StaticClassDecl} ::= \glit{object} $Id$ $ClassBody$
	\alt $AccessMod$ \glit{object} $Id$ $ClassBody$
	\alt \glit{object} $Id$\glit{(}$Param$\glit{)} $ClassBody$
	\alt $AccessMod$ \glit{object} $Id$\glit{(}$Param$\glit{)} $ClassBody$
	
	\gtext{TraitDecl} ::= \glit{trait} $Id$ $ClassBody$
	\alt $AccessMod$ \glit{trait} $Id$ $ClassBody$
	
	\gtext{TraitUniversal} ::= \glit{trait} $TypeId$ \glit{for} $TypeId$ $Scope$
	
	\gtext{ClassBody} ::= \glit{;} | \glit{\{} $ClassMember$ \glit{\}}
	
	\gtext{ClassMember} ::= $ClassMember$ $ClassMember$
	\alt $Id$\glit{(}$Param$\glit{)} $FuncBody$ | $AcessMod$ $Id$\glit{(}$Param$\glit{)} $FuncBody$ %ctor
	\alt $VarDecl$\glit{;} | $AcessMod$ $VarDecl$\glit{;}
	\alt $FuncDecl$ | $ClassDecl$ | $UnionDecl$ | $EnumDecl$
	\alt \glit{event} $TypeId$ $id$\glit{;} | $AccessMod$ \glit{event} $TypeId$ $id$\glit{;}
	
	\newpage	
	
	\gtext{VarDecl} ::= $TypeId$ $Id$ \glit{=} $Expr$ | $StorageMod$ $TypeId$ $Id$ \glit{=} $Expr$
	\alt $TypeId$\glit{[]} $Id$ \glit{=} $Expr$ | $StorageMod$ $TypeId$\glit{[]} $Id$ \glit{=} $Expr$
	\alt $TypeId$\glit{[]} $Id$ \glit{=} $ValueListInitializer$ | $StorageMod$ $TypeId$\glit{[]} $Id$ \glit{=} $ValueListInitializer$
	\alt \glit{var} $Id$ \glit{=} $Expr$ | $StorageMod$ \glit{var} $Id$ \glit{=} $Expr$
	\alt \glit{var} $Id$ \glit{=} $Initializer$ | $StorageMod$ \glit{var} $Id$ \glit{=} $Initializer$
	\alt $TypeId$ $Id$ | $StorageMod$ $TypeId$ $Id$
	\alt $TypeId$\glit{[]} $Id$ | $StorageMod$ $TypeId$\glit{[]} $Id$
	\alt $LambdaType$ $Id$ \glit{=} $LambdaExpr$
	\alt $TupleDecl$

	\gtext{TupleDecl} ::= \glit{(}$ParamType$\glit{)} $Id$ \glit{=} \glit{(}$Argument$\glit{)}
	\alt \glit{(}$Param$\glit{)} \glit{=} \glit{(}$Argument$\glit{)}
	\alt \glit{(}$ParamType$\glit{)[]} $Id$ \glit{=} $Expr$
	\alt \glit{(}$Param$\glit{)[]} $Id$ \glit{=} $Expr$

	\gtext{FuncDecl} ::=  $Id$\glit{(}$FuncParam$\glit{):} $TypeId$ $FuncBody$ % Const-unassignable
	\alt $AccessMod$ $Id$\glit{(}$FuncParam$\glit{):} $TypeId$ $FuncBody$ % Const-unassignable
	\alt $Id$ \glit{= (}$FuncParam$\glit{):} $TypeId$ $FuncBody$  % Reassignable
	\alt $AccessMod$ $Id$ \glit{= (}$FuncParam$\glit{):} $TypeId$ $FuncBody$  % Reassignable
	\alt $AccessMod$ \glit{const} $Id$ \glit{= (}$FuncParam$\glit{):} $TypeId$ $FuncBody$  % Reassignable
	
	\gtext{FuncBody} ::= $Scope$ | \glit{=>} $Expr$
	
	\gtext{FuncParam} ::= $FuncParam$\glit{,} $FuncParam$ | $Param$
	\alt $TypeId$ $Id$ \glit{=} $Literal$	
	
	\gtext{UnionDecl} ::= \glit{union} $Id$ \glit{\{} $UnionMember$ \glit{\}}
	\alt $AccessMod$ \glit{union} $Id$ \glit{\{} $UnionMember$ \glit{\}}
	\alt $AccessMod$ \glit{static union} $Id$ \glit{\{} $UnionMember$ \glit{\}}
	
	\gtext{UnionMember} ::= $UnionMember$ $UnionMember$
	\alt $TypeId$ $Id$\glit{;}	
	
	\gtext{EnumDecl} ::= \glit{enum} $Id$ \glit{\{} $EnumBodyMember$ \glit{\}}
	\alt $AccessMod$ \glit{enum} $Id$ \glit{\{} $EnumBodyMember$ \glit{\}}
	\alt \glit{enum} $Id$\glit{(}$EnumMember$\glit{)\{} $EnumBodyMember$ \glit{\}}
	\alt $AccessMod$ \glit{enum} $Id$ \glit{(}$EnumMember$\glit{)} \glit{\{} $EnumBodyMember$ \glit{\}}
	
	\gtext{EnumBodyMember} ::= $EnumMember$ | $FuncDecl$ | $FuncDecl$ $EnumBodyMember$
	
	\gtext{EnumMember} ::=  $EnumMember$\glit{,} $EnumMember$
	\alt $Id$ | $Id$ \glit{=} $LiteralNoNull$
	
	\newpage
	
	\gtext{Initializer} ::= $ValueListInitializer$ | \glit{\{} $KeyValueListElement$ \glit{\}}
	\alt \glit{\{} $IdValueListElement$ \glit{\}}
	
	\gtext{ValueListInitializer} ::= \glit{\{} $ValueListElement$ \glit{\}}
	
	\gtext{ValueListElement} ::= $Expr$ | $ValueListElement$\glit{,} $ValueListElement$
	
	\gtext{IdValueListElement} :: $Id$ \glit{=} $Expr$ | $IdValueListElement$\glit{,} $IdValueListElement$
	
\gtext{KeyValueListElement} :: \glit{[}$Expr$\glit{]} \glit{=} $Expr$ | $KeyValueListElement$\glit{,} $KeyValueListElement$	
	
	\gtext{Modifier} ::= $StorageMod$ | $AccessMod$ | $TypeMod$ | $CompilerHintMod$ | $\epsilon$
	\alt $AccessMod$ $StorageMod$ | $AccessMod$ \glit{abstract} \glit{variant}
	\alt $AccessMod$ $CompilerHintMod$ | $AccessMod$ $StorageMod$ $CompilerHintMod$
	\alt $AccessMod$ $StorageMod$ $CompilerHintMod$ | $AccessMod$ $TypeMod$
	
	\gtext{StorageMod} ::= \glit{const} | \glit{static} | \glit{override} | \glit{virtual} | \glit{lazy}
	
	\gtext{TypeMod} ::= \glit{variant} | \glit{abstract}
	
	\gtext{CompilerHintMod} ::= \glit{inline} | \glit{final} | \glit{constexpr}
	
	\gtext{AccessMod} ::= \glit{public} | \glit{private} | \glit{protected} | \glit{internal} | \glit{external}
	
	\newpage	
	
	\gtext{Param} ::= $Param$\glit{,} $Param$
	\alt $TypeId$ $Id$ | \glit{const} $TypeId$ $Id$
	\alt $ParamType$ $Id$
	
	\gtext{ParamType} ::= $TypeId$ | $ParameterizedType$ | $ParamType$\glit{,} $ParamType$

	\gtext{LambdaType} ::= \glit{(}$ParamType$\glit{):} $TypeId$
	\alt $TypeId$ \glit{:} $TypeId$
	
	\gtext{ParameterizedType} ::= \glit{\textless}$TypeId$\glit\textgreater{}	
	
	\gtext{Argument} ::= $Expr$ | $Expr$\glit{,} $Expr$
	
	\gtext{Name} ::= $Id$ | $Name$\glit{.}$Name$	
	
	\gtext{BinaryOp} ::= \glit{+} | \glit{-} | \glit{*} | \glit{/} | \glit{\%} | \glit{\textless} | \glit{\textgreater} | \glit{\textless}= | \glit{\textgreater}= | \glit{==} | \glit{!=}
	\alt \glit{\textbar\textbar} | $\&\&$ | \glit{\textbar} | $\&$ | \glit{\textless\textless} | \glit{\textgreater\textgreater} | \glit{=>} | \glit{::} | \glit{??} | \glit{..}
	
	\gtext{UnaryOp} ::= \glit{-} | \glit{!} | \glit{\#} | \glit{++} | \glit{--} | \glit{*}
	
	\gtext{LiteralNoNull}	::= $IntLit$ | $FloatLit$ | $DoubleLit$ | $BoolLit$ | $CharLit$ | $StringLit$
	
	\gtext{Literal}	::= $LiteralNoNull$ | $NullLit$
	
	\gtext{Letter} ::= [\glit{a}-\glit{Z}]
	
	\gtext{Digit} ::= [\glit{0}-\glit{9}]
	
	\gtext{IntLit} ::= $Digit^+$
	
	\gtext{FloatLit} ::= $Digit^+$\glit{.}$Digit^+$\glit{f}
	
	\gtext{DoubleLit} ::= $Digit^+$\glit{.}$Digit^+$

	\gtext{CharLit} ::= \glit{'}$Letter$\glit{'} | \glit{'$\backslash$}$Letter$\glit{'}
	
	\gtext{StringLit} ::= \glit{\"}$(Letter$|$Digit)^*$\glit{\"}	
	
	\gtext{BoolLit} ::= \glit{true} | \glit{false}
	
	\gtext{NullLit} ::= \glit{null}

\end{grammar}
\newpage
\newsection{Operational Semantics}
\newsubsubsection{General Semantics}
\begin{figure}[H]
\centering
	\begin{mathparpagebreakable}
		\infer[VariableLookup]
		{\envall \rho (x) = v \neq (\ell, \omega, \sigma)}
		{\envall x \Rightarrow v, \sigma}
		\and
		\infer[HeapObjectLookup]
		{\envall \rho (x) = (\ell, \omega, \sigma) \rspace \sigma(\ell)=v \rspace \omega = \texttt{O}}
		{\envall x \Rightarrow v, \sigma}
		\and
		\infer[HeapStringLookup]
		{\envall \rho (x) = (\ell, \omega, \sigma) \rspace \sigma(\ell)=v \rspace \omega = \texttt{S}}
		{\envall x \Rightarrow v, \sigma}
		\and
		\infer[HeapArrayLookup]
		{\envall \rho (x) = (\ell, \omega, \sigma) \rspace \sigma(\ell)=v \rspace \omega = \texttt{A}}
		{\envall x \Rightarrow v, \sigma}
		\and
	\end{mathparpagebreakable}
\end{figure}\noindent
\newsubsubsection{Classes}
\newsubsubsection{Objects}
\newsubsubsection{Structs}
\newsubsubsection{Union}
\newsubsubsection{Enum}
\newsubsubsection{Traits}
\newpage
\newsection{Type Semantics}
\begin{figure}[H]
\centering
\framebox{
	\parbox[t][1.9 cm]{4.2 cm}{
	\addvspace{-0.25cm}
		\begin{align*}
			\theta &=TypeEnv \\
			\gamma &=TypeLookupEnv \\
			\eta &=ReferenceTypeEnv
		\end{align*}
	}
}
\framebox{
	\parbox[t][1.9cm]{4.2 cm}{
	\addvspace{-0.25cm}
		\begin{align*}
			\theta, \gamma, \eta &\vdash Expr \ \texttt{:} \ Type \\
			\theta, \gamma, \eta &\vdash Decl \ \texttt{:} \ \theta, \gamma \\
			\theta, \gamma, \eta &\vdash Decl \ \texttt{:} \ \theta, \gamma, \eta
		\end{align*}
	}
}
\framebox{
	\parbox[t][1.9cm]{4.2 cm}{
	\addvspace{-0.25cm}
		\begin{align*}
			\subtype{\texttt{int}}{\subtype{\texttt{float}}{\subtype{\texttt{double}}{\texttt{INumeric}}}}  \\ 
		\end{align*}
	}
}
\framebox{
	\parbox[t][1.9cm]{7.4 cm}{
	\addvspace{-0.15cm}
		\begin{equation*}			
			\texttt{typeof}(t, \gamma, \eta) = \begin{cases} 
            \texttt{Ref}(\gamma(t)) & t \in \gamma, \gamma(t) \in \eta  \\
            \gamma(t) & t \in \gamma, \gamma(t) \notin \eta \\
            t & \text{otherwise}\footnote{}            
        	\end{cases} 
		\end{equation*}
	}
}
\framebox{
	\parbox[t][1.9cm]{7.4 cm}{
	\addvspace{0.35cm}
        \begin{equation*}			
			\texttt{base}(\tau_1, \tau_2) = \begin{cases} 
            \tau_1 & t_2 <: t_1  \\
            \tau_2 & t_1 <: t_2     
        	\end{cases}
		\end{equation*}
	}
}
\end{figure}
\footnotetext{Atomic type, such as \texttt{int}, \texttt{bool} or \texttt{char}.}
Additionally, we note that $\theta$ is local to the expression while $\gamma$ and $\eta$ are global environments\footnote{With respect to current domain as elements in $\gamma$ may have local definitions not globally visible.}. Additionally, $\eta \subset \gamma$ such that no element in $\eta$ can be an atomic type and must be a type that is defined during compile-time. 
Another thing to note is $\tau$ consists of the tuple $(\phi, \mu)$. Where $\phi$ is the set of all fields belonging to the type. Unless it's an atomic type, in which case this will be the empty set. $\mu$ is the set of all methods.% Additionally, there's the implication $t \in \phi \iff t \notin \mu$. Meaning some type identifier t can only be in one of the two (or neither).
\begin{figure}[H]
\centering
	\begin{mathparpagebreakable}
		\infer[T-IntLit]
		{i\in \mathbb{N}}
		{\typemodel i \typeis{\texttt{int}} }
		\and
		\infer[T-Identifier]
		{\tau=\theta(id) \rspace id \in \theta}
		{\typemodel id \typeis{\tau}}
		\and
		\infer[T-Addition]
		{\typemodel e_1 \ \texttt{:} \ \tau_1 \rspace \typemodel e_2 \ \texttt{:} \ \tau_2 \rspace \tau'=\texttt{base}(\tau_1, \tau_2) \rspace \tau' <: \texttt{INumeric}}
		{\typemodel e_1 \ \texttt{+} \ e_2 \typeis{\tau'}}
		\and
		\infer[T-Subtraction]
		{\typemodel e_1 \ \texttt{:} \ \tau_1 \rspace \typemodel e_2 \ \texttt{:} \ \tau_2 \rspace \tau'=\texttt{base}(\tau_1, \tau_2) \rspace \tau' <: \texttt{INumeric}}
		{\typemodel e_1 \ \texttt{-} \ e_2 \typeis{\tau'}}
		\and
		\infer[T-Multiplication]
		{\typemodel e_1 \ \texttt{:} \ \tau_1 \rspace \typemodel e_2 \ \texttt{:} \ \tau_2 \rspace \tau'=\texttt{base}(\tau_1, \tau_2) \rspace \tau' <: \texttt{INumeric}}
		{\typemodel e_1 \ \texttt{*} \ e_2 \typeis{\tau'}}
		\and
		\infer[T-Division]
		{\typemodel e_1 \ \texttt{:} \ \tau_1 \rspace \typemodel e_2 \ \texttt{:} \ \tau_2 \rspace \tau'=\texttt{base}(\tau_1, \tau_2) \rspace \tau' <: \texttt{INumeric}}
		{\typemodel e_1 \ \texttt{/} \ e_2 \typeis{\tau'}}
		\and
		\infer[T-DeclVar]
		{\typemodel e \ \texttt{:} \ \tau \rspace \theta'=\theta[x \mapsto \tau'] \rspace \tau'=\texttt{typeof}(t,\gamma, \eta) \rspace \tau <: \tau'}
		{\typemodel t \ x \ \texttt{=} \ e \typeis{\theta', \gamma, \eta}}
		\and
		\infer[T-NewObject]
		{\tau = \texttt{typeof}(t,\gamma, \eta) \rspace \typemodel e_1,\dots, e_n \typeis{\tau_1,\dots,\tau_n}}
		{\typemodel \texttt{new} \ t \texttt{(} e_1,\dots,e_n \texttt{)} \typeis{\tau}}
		\and
		\infer[T-FieldAccess]
		{\typemodel e \ \texttt{:} \ \tau \rspace \tau=(\phi, \mu) \rspace \tau'=\phi(id) \rspace id\in\phi}
		{\typemodel e\texttt{.}id \typeis{\tau'}}
		\and
		\infer[T-MethodAccess]
		{\typemodel e \ \texttt{:} \ \tau \rspace \tau=(\phi, \mu) \rspace \tau'=\mu(id) \rspace id\in\mu}
		{\typemodel e\texttt{.}id \typeis{\tau'}}
		\and
	\end{mathparpagebreakable}
\end{figure}\noindent
When inferring the type of a scope - the whole set of control paths must be considered. Additionally, the last expression of a scope is returned to the calling scope.
\newpage
\newsection{Compilation Semantics}
\newpage
\newsection{Bytecode Semantics}
\end{document}